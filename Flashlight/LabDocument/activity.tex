%describe the tasks to be completed that will satisfy the given objective
\section{Activity}

\subsection{Research}
Read through the Hello World tutorial at http://developer.android.com/resources/tutorials/hello-world.html and any other tutorial you might be interested in. 
The Android framework fundamentals is also good information on the basic components of an Android application. http://developer.android.com/guide/topics/fundamentals.html.

We will be working with the \verb=Camera= class because the LED flashlight is part of the camera. 
Look at the \verb=Camera= class, specifically \verb=Camera.open()=,\verb=Camera.getParameters()=, \verb=Camera.setParameters()=, \verb=Camera.startPreview()=, and \verb=Camera.stopPreview()=. http://developer.android.com/reference/android/hardware/\verb=Camera=.html.
We will also be using the \verb=Camera.Parameters= class using the \verb=Camera.Parameters.setFlashMode()= method. http://developer.android.com/reference/android/hardware/Camera.Parameters.html\#setFlashMode(java.lang.String).

\subsection{Explore}
Open the FlashlightActivity.java file and familiarize yourself with the code provided. 
Read all the comments and note the parts of the code you will be writing. 
The parts you will be writing will be clearly marked.

\subsection{Import}

\subsection{Flashlight Off Method}
First, locate the flashlightClicked method of the FlashlightActivity class. 
This method will run whenever you touch the screen. 
It gets called by the onTouch method. 
More on the onTouch method later. 
Back in the flashlightClicked method, notice the if statement. 
If the flashlight is on, it runs the turnOffFlashlight method. 
This method is empty and will be created by you. 
Likewise, if the flashlight is off, it runs the turnFlashlightOn method. 
You will create this as well.

Inside of the turnOffFlashlight method, first you will need to set the flash mode to off. 
Do this by calling the setFlashMode() method of camParams. 
You will pass it a string of value “off”. 
Next, we will pass the camParams object to the setParameters() method of the camera class variable. 
Finally, you will close the camera preview that was started in turnOnFlashlight(). 
To do this, call the stopPreview() method of the camera class variable.

\subsection{Flashlight On Method}
Copy and paste “Off Method” code, change little things.

\subsection{OnTouch Method}
Locate the onTouch method inside the FlahslightActivity class. 
You will edit this method to check if the MotionEvent parameter is ‘touch down’ action. 
To acopmlish this, call the getAction() method of the MotionEvent parameter and check if this is equal to the MotionEvent.ACTION\_DOWN integer. 
If these are equal, you will call the flashlightClicked() method and pass it the View variable as the parameter.


