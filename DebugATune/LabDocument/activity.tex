%describe the tasks to be completed that will satisfy the given objective
\section{Activity}
This section describes the tasks to be completed that will satisfy the given objectives.
\subsection{Background Information}
This section provides some information that will help you complete or understand the contents of the lab.

Errors can be separated into multiple classifications depending on their source. 
Compiler Errors are syntactical errors in code. The most common cause of this is a typo. 
Linker Errors are caused when a file or package referenced in one class cannot be located by the compiler.
Runtime Errors occur when there is a logical error in a program. 
This can be as simple as forgetting to break a loop, resulting in an infinite loop, or as complex as accessing a memory location beyond that allocated to the program.

\subsection{Implementation}
This section will guide you to complete the lab.
You have been provided an Eclipse project named Error Management.
Follow the directions outlined below.

To complete this lab successfully you will need to:
\begin{itemize}
\item Identify and remove all compiler Errors
\item Identify and remove all linker Errors
\item Identify and remove all runtime Errors
\end{itemize}
For each task, you must record the line number, erroneous code fragment, error type, an explanation of why it is in error, and the fixed code fragment. For runtime errors you are required to submit the steps you took to debug the problem.