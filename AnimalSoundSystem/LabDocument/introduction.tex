%describe the background necessary to understand the lab's content, with some reservation (i.e., not too much detail)
%possibly provide walkthrough for simple problem
\section{Introduction}

In this lab you will be learning about Inheritance and polymorphism. This will be accomplished by building an Animal sound system.To implement this lab, the pictures and sounds will be provided and the students will have to map each picture with the sound utilizing the idea of inheritance and polymorphism. 

Inheritance can be described as compile-time mechanism in Java that enables you to extend a class (called the base class or superclass) with another class (called the derived class or subclass). In Java, inheritance is utilized for two purposes, class inheritance and interface inheritance. In this lab we will be focusing mostly on class inheritance. The idea of inheritance really is easy but powerful, When you intend to create a new class and there exits a class already that includes some of the code that you want, you can derive your new class from the existing class. In doing this, you can reuse the fields and methods of the existing class without having to write (and debug!) them yourself. 

In programming languages, polymorphism implies that some code or operations or objects act otherwise in different contexts. The dictionary definition of polymorphism means a principle in biology in which an organism or species may have many different forms or phases. This principle can also be utilized on object oriented programming and languages like the Java language. Subclasses of a class can define their own unique behaviors and yet share some of the same functionality of the parent class.


